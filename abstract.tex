\begin{abstract}
% The abstract serves as an executive summary of your project.
% Your abstract should cover at least the following topics, 1-2 sentences for
% each: what area you are in, the problem you focus on, why existing work is
% insufficient, what the high-level intuition of your work is, maybe a neat
% design or implementation decision, and key results of your evaluation.

GraalVM Native Image compiles Java bytecode into a native image that operates under the closed-world assumption. This means that every reflectively-accessed element such as Class, Executable, or Field must be provided as reachability metadata to the image-build process. In Native Image, the reachability metadata is either 1) provided via user-provided configuration files in JSON format or 2) computed by partially evaluating the input program to pre-compute the reflectively-accessed elements.

Due to the long image-build times, creating the correct metadata is a time-consuming process for users: if the metadata for any element is missing, the entire image must be rebuilt. The objective of this project is to allow users to compute metadata with a quick turnaround by introducing a new mode to Java, so that it operates under the closed-world assumption and behaves exactly like Native Image.

We modify all reflective Java features to operate under the closed-world assumption by checking the reachability metadata to determine if the reflectively accessed element is included in the image. We introduce a Tracing Agent to facilitate the collection of the metadata. 
%% TODO check on the bytecode-to-bytecode partial evaluator move to future work
% To pre-compute the reflectively-accessed elements we implement a bytecode-to-bytecode partial evaluator that transforms classes before they are loaded and extracts reachability metadata from them. 
Finally, implementing the Java mode enables us to drive Native Image specifications and to clearly state what the expected behaviour.   
\end{abstract}

