\chapter{Related Work}

%The related work section covers closely related work. Here you can highlight
%the related work \cite{DBLP:conf/fmcad/KuncakH21, DBLP:conf/pldi/GveroKKP13}, how it solved a problem, and why it solved a different
%problem \cite{DBLP:conf/nsdi/YabandehKKK09}. Do not repeat at length the content of related work, but rather focus
%on how it compares the present work: by enabling it, by exploring a slightly different approach \cite{DBLP:conf/cav/PiskacK08}, or by being subsumed by the current work.
%Do not play down the importance of related work. Say what is different and how
%you overcome some of the weaknesses of related work by discussing the 
%trade-offs. Stay positive and factual: do not use subjective assessment, but point out real differences.
%This section is usually 3-5 pages.
%Please use bibtex to enter publications. To find a reference, use "DBLP PaperTitle" in a search engine and then use the bib file provided by bibtex.
%Feel free to use wikipedia to find related references, but do not cite wikipedia pages themselves. Consult those pages and your advisors to find the original sources and suggestions for background material.
%%%%%%%%%%%%%%%%%%%%%%%%%%%%%%%%%%%%%%%%%%%%%%%%%%%%%%%%%%%%%%%%%%%%%%%%%%%%%%%%%%%%%%%%%%%%%%%%%%%

%%%%%%%%%%%%%%%%%%%%%%%%%
\section{Combining dynamic and static compilers}
%%%%%%%%%%%%%%%%%%%%%%%%%
Quicksilver~\cite{serrano_quicksilver_2000} is a quasi-static compiler for Java, that relies on both static and dynamic compilation. 
When a Java class is compiled, a quasi-static image~(QSI) is created. At runtime, the image is adapted such that the JVM may reuse it. If the QSI for the class does not exist, the compiler falls back to dynamic compilation to load the class. 
By first starting in a closed-world assumption and expanding to a dynamic environment when needed, this approach tackles our problem from the other direction.
Moreover, "stitching", the process used to adapt the images at runtime is quite complex. In contrast, one of the requirement for Java under native restrictions is to introduce a minimal set of changes to Java.

Another project that attempts to bridge the gap between static and dynamic compilation is Microsoft .NET Common Language Runtime~\cite{noauthor_common_2023}, an implementation of the Common Language Infrastructure~(CLI)~\cite{noauthor_common_2012}. 
Similarly to Native Image, NGen~\cite{noauthor_clr_2019} pre-compiles executables into native binaries, and resolves all dependencies at build time.
Metadata for executables are emitted by the compilers. At runtime, the metadata is checked to dynamically load classes, resolve method invocation, etc., and if needed .NET can fall back to JIT compilation.

%%%%%%%%%%%%%%%%%%%%%%%%%
\section{Reflection}
%%%%%%%%%%%%%%%%%%%%%%%%%
Reflection is an integral part of the design of Smalltalk-80~\cite{goldberg_smalltalk-80_1983}. In Smalltalk-80 everything is an object. Classes, metaclasses, and methods are all objects. Since the objects contain all the necessary metadata for reflection, the objects can be directly queried, such that instances of classes and classes themselves can be accessed and modified at runtime. 

Swift's proposal for opt-in reflection metadata~\cite{noauthor_pitch_2022} is a way to mitigate the problem that can arise because of the lack of transparency of the Reflection API, and the inclusion of unconsumed metadata in binaries. In Swift, reflection can be turned off on a module basis, which can lead to inconsistent behaviours at runtime if another module's implementation depended on reflection. Moreover, reflective calls cannot be statically analysed, and binaries can end up being polluted with unused metadata as a result. Instead of parsing a JSON configuration, Swift introduce a marker protocol called \verb|Reflectable|. This marker is checked by the compiler, so warning can be output when a module depends on the metadata, and metadata is emitted only when it is also consumed. The protocol also has the advantage of making reflection more transparent to API developers.

Like Native Image and the Java mode, Jax~\cite{tip_practical_1999} relies on a configuration file to specify which element will be reflectively accessed at runtime. They also offer an instrumented application to trace and output the metadata.

Scala Native~\cite{noauthor_scala_nodate} is an AOT compiler and a lightweight runtime for Scala that uses annotation to support reflection. Modules and classes marked with the \verb|@EnableReflectiveInstantation| can be reflectively queried at runtime by other libraries and frameworks.
This approach is similar to what we plan to implement, as this would offer the advantage of being statically analysable and more transparent for users.

%%%%%%%%%%%%%%%%%%%%%%%%%
\section{Porting programming languages}
%%%%%%%%%%%%%%%%%%%%%%%%%
`C (Tick C)~\cite{engler_c_1996} is an example of porting a static language, namely ANSI C, to a dynamic language. This allows dynamic code generation for better code optimization, but also a new kind of application like JIT compilers for C. Poletto, Hsieh, Engler and Kasshoek's tcc~\cite{poletto_c_1999} is an implementation of such a compiler. It consists of two backend, a static and a dynamic.
The former compiles the usual C or non dynamic part expressions, while the latter generates dynamic code that the static backend will then compile.

JavaScript strict mode as defined in the ECMAScript® 2024 Language Specification~\cite{noauthor_ecmascript_nodate}, is implemented along side the usual "sloppy mode", and applies another set of semantics to the language. The changes impose stricter variable naming and declaration, stricter expression scope, prohibits hoisting, and throws exception instead of silent failing. Similarly to the Java mode, Google's V8 JavaScript VM implementation~\cite{noauthor_v8_nodate}, relies on a flag system, and a "use\_strict" annotation on methods and blocks of code, to apply checks where the behaviour is expected to diverge from the sloppy mode. The main difference with our approach is that this set of semantics does not deal with dynamic features like class loading, as such no differentiation need to be made between runtime and compile time behaviour. 