%%%%%%%%%%%%%%%%%%%%%%%%%%%%%%
\chapter{Background}
%%%%%%%%%%%%%%%%%%%%%%%%%%%%%%

% The background section introduces the necessary background to understand your
% work. This is not necessarily related work but technologies and dependencies
% that must be resolved to understand your design and implementation.
% This section is usually 3-5 pages.

%%%%%%%%%%%%%%%%%%%%%%%%%%%%%%
\section{The Java Virtual Machine}
%%%%%%%%%%%%%%%%%%%%%%%%%%%%%%
Following the "write once, run anywhere" adage, the typical way of running a Java application is on a JVM.
The JVM offers an additional, target-independent, layer of abstraction between implementations of the Java programming language~\cite{noauthor_java_nodate-1} and the operating system itself.
The JVM is responsible for interpreting and executing Java bytecode, and provides four key components~\cite{dannarapu_jvm_2023-1}: 
the class loader to load classes into memory, the runtime data area (that is, the method area, heap, and stack) to store program data, the execution engine, which reads and executes bytecode instructions, and the garbage collector. The JVM Specifications~\cite{noauthor_java_nodate-2} describe in detail the requirements for a JVM to be compliant. 
The thesis targets a particular implementation, mainly Oracle's Java HotSpot~\cite{noauthor_hotspot_nodate} and its JIT compiler for Java 21. 

HotSpot relies on tiered compilation to optimize code execution. It consists of an interpreter and two JIT compilers: C1 and C2. 
C1 compiles bytecode with a minimal time and space overhead, applying only a limited set of optimizations, while C2 employs a more aggressive optimization strategy, requiring more resources.
The VM starts by interpreting the bytecode to minimize startup time, and collects profiling information at the same time. Using this information, once a method is deemed "hot" enough, C1 kicks in and starts compiling it. When the number of times the method has been invoked (or the number of back-edges to that method) crosses another threshold, C2 recompiles the method into a more optimized form. If an optimization is proven wrong, the method is deoptimized and the compiler reverts back to interpretations.  

%%%%%%%%%%%%%%%%%%%%%%%%%%%%%%
\section{GraalVM}
%%%%%%%%%%%%%%%%%%%%%%%%%%%%%%
GraalVM is a runtime ecosystem developed by Oracle Labs, and features three core components: the Graal compiler~\cite{noauthor_graal_nodate}, Truffle~\cite{noauthor_truffle_nodate}, and Native Image~\cite{wimmer_initialize_2019}.
Graal is a compiler that relies on multiple optimizations phases to produce optimized machine code from bytecode. The compiler operates on a language-agnostic intermediate representation called Graal~IR~\cite{duboscq_graal_2013}. Graal~IR not only facilitates the implementation of speculative optimizations, it also enables Graal to compile programming languages that are not JVM-based (e.g., JavaScript, Python, Ruby). % and run them on the same JVM.
The JVMCI~\cite{noauthor_jep_nodate}~(JVM Compiler Interface) API offers mechanisms to access the JVM internal data structures and install compiled code. Through this interface, Graal can integrate as a JIT compiler into HotSpot's system and replace C2. 
Graal can also be integrated with Native Image as the AOT compiler.

Native Image is an image builder that compiles Java bytecode ahead of time into a standalone binary. This approach aims at reducing the startup time and memory footprint of Java applications. Native Image operates under the closed-world assumption: all Java classes, which includes the JDK, must be known at build time to be included in the image.
This also means, that every reflectively-accessed elements must be specified in the reachability metadata to be made available to the build process. 
At build time, during static analysis, points-to-analysis and heap snapshotting are iteratively applied so that only reachable code remains in the final image. Additionally, part of the initialization code of the application can be executed at build time rather than at runtime, thus improving the startup time.

%%%%%%%%%%%%%%%%%%%%%%%%%%%%%%
\section{Java's Dynamic Features}
%%%%%%%%%%%%%%%%%%%%%%%%%%%%%%
One of the cornerstones of the Java platform is its dynamic features. At run time, JARs, resources, and proxies can be loaded, and methods can be invoked without the compiler having seen them before. In the following section, we will dive deeper into the mechanisms of dynamic class loading and invocation.

%%%%%%%%%%%%%%%%%%%%%%%%%%%%%%
\subsection{Dynamic Class Loading}\label{dynamic_class_loading}
%%%%%%%%%%%%%%%%%%%%%%%%%%%%%%
Dynamic class loading~\cite{liang_dynamic_1998} allows Java applications to load, link, and initialize classes at runtime. Plugins, for example, can be loaded and unloaded without restarting the application; dependency injection relies on late binding of dependencies, which is made possible with dynamic class loading. 

Class loaders are the underlying mechanism used for loading classes. 
There are two types of class loaders: the bootstrap (or system) class loader, which loads all the system classes, and user-defined class loaders, which load classes from user-defined sources.
Class loaders operate on a delegation model. If a class loader \verb|L| directly loads \verb|C|, then \verb|L| is the \emph{defining loader} of \verb|C|, otherwise \verb|L| recursively delegate the loading to another class loader, until the defining loader of \verb|C| is found. In this case, \verb|L| is an \emph{initiating loader} of \verb|C|. Concretely, a class loader in an instance of a subclass of the class \verb|ClassLoader|. 
Figure~\ref{fig:classloader}, shows the key methods of the class.

\begin{figure}[ht]
    \centering
\begin{lstlisting}[language=Java]
class ClassLoader {
    public Class<?> loadClass(String name);
    protected final Class<?> findLoadedClass(String name);
    static native Class<?> defineClass1(ClassLoader loader, String name, 
                                        byte[] b, int off, int len,
                                        ProtectionDomain pd, String source);
    static native Class<?> defineClass2(ClassLoader loader, String name, 
                                        java.nio.ByteBuffer b,
                                        int off, int len, ProtectionDomain pd,
                                        String source);
    static native Class<?> defineClass0(ClassLoader loader,
                                        Class<?> lookup,
                                        String name,
                                        byte[] b, int off, int len,
                                        ProtectionDomain pd,
                                        boolean initialize,
                                        int flags,
                                        Object classData);
    public static ClassLoader getSystemClassLoader();
    ...
}
\end{lstlisting}
    \caption{Core methods of the \texttt{ClassLoader} class}
    \label{fig:classloader}
\end{figure}


The method \verb|loadClass| is the usual entry point to initiate class loading. It can be called directly from user code, or triggered by a JVM instruction.
When loading a class \verb|C| with the system class loader (see Figure~\ref{fig:class_C}), the JVM first checks with the method \verb|findLoadedClass|, if \verb|C| has already been loaded, in which case it immediately returns the class, otherwise, the class loader attempts to locate the corresponding data. 

Linking a class or interface \verb|C| is done in three steps: (1) the JVM verifies that the binary representation of the class or interface \verb|C| is correct, (2) the static fields of \verb|C| are created and initialized, and (3) symbolic references are resolved into direct references. If \verb|C| has a direct superclass or superinterface \verb|D|, then \verb|D| must also be verified and prepared. 

Once all symbolic references have been resolved, the JVM invokes the method \verb|defineClass1|, \verb|defineClass2| or \verb|defineClass0| to transform the array of bytes into a run-time representation of the class, and returns a \verb|Class| object.

\begin{figure}[ht]
    \centering
\begin{lstlisting}[language=Java]
class C extends D {
}
class Main {
    public static void main(String[] args) {
        ClassLoader cl = ClassLoader.getSystemClassLoader();
        Class<?> c = cl.loadClass("C");
    }
}
\end{lstlisting}
    \caption{Dynamically loading the class \texttt{C} at runtime.}
    \label{fig:class_C}
\end{figure}

%%%%%%%%%%%%%%%%%%%%%%%%%%%%%%
\subsection{The \texttt{invokedynamic} Instruction}
%%%%%%%%%%%%%%%%%%%%%%%%%%%%%%
The \verb|invokedynamic| instruction~\cite{noauthor_java_nodate} was introduced to support the invocation of methods without static type information at runtime.
This instruction is distinct from the JVM's traditional method invocation instructions (\verb|invokevirtual|, \verb|invokeinterface|, \verb|invokestatic|, and \verb|invokespecial|) in that it does not bind to a method at compile time. Instead, it defers the decision of which method to invoke until runtime.

At the core of the \verb|invokedynamic|'s resolution mechanism is the bootstrap method. A bootstrap method is a user-defined method that the JVM invokes the first time an \verb|invokedynamic| instruction is executed. Some bootstrap methods are provided by the JVM itself. For example, lambda expressions in Java use the \verb|LambdaMetafactory#metafactory| or \verb|altmetafactory| as their bootstrap methods.
This bootstrap method returns a \verb|CallSite| object, which encapsulates a target \verb|MethodHandle|. The \verb|MethodHandle| is a reference to an executable or a field. 
It contains mechanisms to invoke or access the object it references, as well as a reference to its symbolic form: a \verb|java.lang.invoke.LambdaForm|. The \verb|LambdaForm| has a \verb|MemberName| field that encapsulates the element's declaring class, name, and its type parameters if it has any. This \verb|MemberName| must be resolved before the method handle can be invoked. 
% The \verb|MethodHandle| is a reference to an executable or a field and contains all the necessary mechanisms and data structure to invoke or access the object it references (e.g., the element's defining class, name, type signature, and methods to access it).
Once resolved, the \verb|CallSite| is cached and associated to a particular \verb|invokedynamic| instruction.
For subsequent invocations, the bootstrap method is not re-executed, instead the target method handle of the \verb|CallSite| can be directly invoked.
% Before the \verb|CallSite| is returned, it must first be resolved. \verb|CallSite| resolution involves three steps: (1) the bootstrap method handle is resolved, meaning all symbolic references to classes, interfaces, fields, and methods, as well as, all classes and interfaces contained in the method descriptor must be resolved, (2) the bootstrap method is invoked, and finally (3) the \verb|CallSite| is verified.
%%%%%%%%%%%%%%%%%%%%%%%%%%%%%%
\section{Java Reflection API}
%%%%%%%%%%%%%%%%%%%%%%%%%%%%%%
Java Reflection provides an API that allows programs to dynamically access classes, their executables and fields, their metadata, as well as invoke methods, and create instances of classes at runtime, with no compile-time dependency. It is also the core mechanism used to implement serialization and annotation processing. Figure~\ref{fig:reflective_calls} shows an example of a Java program that uses the Reflection API to invoke a method.

\begin{figure}[ht]
    \centering
\begin{lstlisting}[language=Java]
class Greeter {
    public static void greet(String message) { 
        System.out.println(message); 
    }
}
public class Main {
    public static void main(String[] args) throws ReflectiveOperationException {
        Class<?> greeter = Class.forName("Greeter");
        Method greet = greeter.getDeclaredMethod("greet", String.class);
        greet.invoke(greeter, "Hello there!");
    }   
}
\end{lstlisting}
    \caption{Through reflection, the program obtains a \texttt{Class} object of type \texttt{Greeter} and invokes its method \texttt{greet} at runtime.}
    \label{fig:reflective_calls}
\end{figure}

The implementation of reflection in HotSpot relies on the pointer contained in every Java object header that points to the metadata for its declaring class~\cite{evans_ben_reflection_nodate}. At runtime, the JVM traverses the pointers until it reaches the object's declaring class and returns the reflectively-accessed element.
% Implementing reflection in HotSpot is rather straightforward, as every Java object header contains a pointer to the metadata for its declaring class. At runtime, the JVM traverses the pointers until it reaches the object's declaring class~\cite{evans_ben_reflection_nodate}, and returns the queried element.

%%%%%%%%%%%%%%%%%%%%%%%%%%%%%%
\section{Native Image's Implementation of Reflection and Java's Dynamic Features}
%%%%%%%%%%%%%%%%%%%%%%%%%%%%%%
Reflection, dynamic class loading and dynamic invocation all require runtime information that Native Image does not have at build time. Therefore, reflectively-accessed element must be provided as reachability metadata, which can be user-provided in the form of a JSON reflection configuration file.  
Figure~\ref{fig:reflective_calls_invoke}, for example, would require the entry in Figure~\ref{fig:reflect_config} to be specified in a \verb|META-INF/native-image/reflect-config.json| file placed at the root of the project.

\begin{figure}[ht]
    \centering
\begin{lstlisting}[language=Java]
public class Main {
    public static void main(String[] args) throws ReflectiveOperationException 
    {
        invoke("Greeter", "greet", String.class, "Hello there!");
    }   
    public static void invoke(String className, 
        String methodName, Class<?> paramType, String arg) 
        throws ReflectiveOperationException 
    {
        Class<?> clazz = Class.forName(className);
        Method m = clazz.getDeclaredMethod(methodName, paramType);
        m.invoke(clazz, arg);
    }
}
\end{lstlisting}
    \caption{The \texttt{main} method invokes, through reflection, the method \texttt{greet} of the \texttt{Greeter} class. The \texttt{className} and \texttt{methodName} passed to the method \texttt{invoke} are not passed as constants and require to be registered for reflection.}
    \label{fig:reflective_calls_invoke}
\end{figure}

\begin{figure}[ht]
    \centering
\begin{lstlisting}
[{
    "name": "Greeter",
    "methods": [{"name": "greet", parameterTypes: ["java.lang.String"]}]
}]    
\end{lstlisting}
    \caption{Reflection configuration required to access the class \texttt{Greeter} and invoke its method \texttt{greet} with a \texttt{String} argument.}
    \label{fig:reflect_config}
\end{figure}

To help capture this metadata, GraalVM provides a Tracing Agent. Through the JVM Tool Interface~\cite{noauthor_jvmtm_nodate}, a native library, or \emph{agent}, can attach to the JVM and access and control the state of the applications running on the JVM. GraalVM's Tracing Agent introduces a list of breakpoints for each method of the Java Reflection API to interrupt the application's normal execution flow and record the reflectively-accessed elements instead. When the execution run has finished, the agent outputs a JSON reflection configuration that can be passed to the image-build process.

If the reflectively-accessed element is passed as a constant or effectively constant, as in Figure~\ref{fig:reflective_calls}, the reachability metadata can be pre-computed. The constant gets constant-folded at image build time, and the element does not need to be manually included in the JSON reflection configuration.

There are two outcomes to dynamically loading a class in Native Image: (1) the argument passed to \verb|loadClass| is a constant or the class is included in the reachability metadata and the call succeeds, (2) the class is not a constant, nor is it registered, in which case the image will throw a missing registration error at runtime.

For the \verb|invokedynamic| instruction, the main difference between HotSpot's implementation and Native Image's resides in the bootstrap method resolution. Native Image distinguishes how it handles the resolution based on whether the bootstrap method is trusted, that is if it comes from the Java language or not. For trusted bootstrap methods, resolution occurs at build time. Otherwise, the bootstrap method is interpreted at runtime. The resolution relies on reflection, and requires the declaring class, the method, and the types in the method signature to be registered for reflection.