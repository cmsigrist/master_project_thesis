%%%%%%%%%%%%%%%%%%%%%%%%%%%%%%%%
\chapter{Specs}
%%%%%%%%%%%%%%%%%%%%%%%%%%%%%%%%
JDK Behaviour under Native restrictions

%%%%%%%%%%%%%%%%
\section{java.lang.SecurityManager}
%%%%%%%%%%%%%%%%
Under the native restriction, the java.lang.SecurityManager behaves differently:
\begin{enumerate}
    \item java.lang.System\#getSecurityManager() always returns null.
    \item java.lang.System\#setSecurityManager(SecurityManager sm) always throws an UnsupportedOperationException. 
\end{enumerate}

%%%%%%%%%%%%%%%%
\section{Class Loading, Linking and Initialization}
%%%%%%%%%%%%%%%%
When running under native restrictions the following rules apply:
\begin{enumerate}
    \item The application class loader can not be overridden by the user.
    \item All classes from the classpath are linked before the execution of the program.
    \item All invokedynamic instructions are resolved in the step before program execution and replaced with adequate bytecodes. For example, all calls to java.lang.invoke.LambdaMetafactory are replaced with the calls to a pregenerated class.
\end{enumerate}

The native restrictions require that java.lang.ClassLoader\#defineClass0, java.lang.ClassLoader\#defineClass1, java.lang.ClassLoader\#defineClass2 always throw an UnsupportedOperationException, if the class was not preloaded. 

\subsection{Resolving classes}
%The Java Virtual Machine instructions anewarray, checkcast, getfield, getstatic, instanceof, invokedynamic, invokeinterface, invokespecial, invokestatic, invokevirtual, ldc, ldc\_w, multianewarray, new, putfield, and putstatic make symbolic references to the run-time constant pool. Execution of any of these instructions requires resolution of its symbolic reference.
%And 5.4.3.1. Class and Interface Resolution
%https://docs.oracle.com/javase/specs/jvms/se7/html/jvms-5.html#jvms-5.4.3.1
%We can say that defineClass is forbidden unless it is used as a part of the class and interfeace resolution of all of these instructions.
%Also, we can request that all invokedynamic instructions are pre-resolved.
Many Java Virtual Machine instructions - \textit{anewarray}, \textit{checkcast}, \textit{getfield}, \textit{getstatic}, \textit{instanceof}, \textit{invokedynamic}, \textit{invokeinterface}, \textit{invokespecial}, \textit{invokestatic}, \textit{invokevirtual}, \textit{ldc}, \textit{ldc\_w}, \textit{ldc2\_w}, \textit{multianewarray}, \textit{new}, \textit{putfield}, and \textit{putstatic} - rely on symbolic references in the run-time constant pool. Execution of any of these instructions requires *resolution* of the symbolic reference. 


     * Java Virtual Machine instructions {@code anewarray}, {@code checkcast}, {@code getfield}, {@code getstatic},
     * {@code instanceof}, {@code invokedynamic}, {@code invokeinterface}, {@code invokespecial}, {@code invokestatic},
     * {@code invokevirtual}, {@code ldc}, {@code ldc_w}, {@code multianewarray}, {@code new}, {@code putfield}, and
     * {@code putstatic} make symbolic references to the run-time constant pool.
     * Under native restrictions, calls to
     * {@link ClassLoader#defineClass0(ClassLoader, Class, String, byte[], int, int, ProtectionDomain, boolean, int, Object)},
     * {@link ClassLoader#defineClass1(ClassLoader, String, byte[], int, int, ProtectionDomain, String)}, and
     * {@link ClassLoader#defineClass2(ClassLoader, String, ByteBuffer, int, int, ProtectionDomain, String)}
     * result in {@link UnsupportedOperationException}, unless it is used as a part of the class and interface
     * resolution of the aforementioned instructions. If another instruction attempts to load a class at runtime, the call
     * will throw an {@link java.util.MissingReflectionRegistrationError}, unless the class was preloaded.
     * Loading hidden class at runtime always results in an {@link UnsupportedOperationException} being thrown.
     *


\textit{Loading} refers to the process of finding the binary form of a class or interface with a particular name, perhaps by computing it on the fly, but more typically by retrieving a binary representation previously computed from source code by a Java compiler, and constructing, from that binary form, a `Class` object to represent the class or interface

(Loading) The loading process is implemented by the class `ClassLoader` and its subclasses. The method `defineClass` of class `ClassLoader` may be used to construct `Class` objects from binary representations in the `class` file format

(Linking) Resolution: The binary representation of a class or interface references other classes and interfaces and their fields, methods, and constructors symbolically, using the binary names ([§13.1](https://docs.oracle.com/javase/specs/jls/se21/html/jls-13.html\#jls-13.1)) of the other classes and interfaces. For fields and methods, these symbolic references include the name of the class or interface of which the field or method is a member, as well as the name of the field or method itself, together with appropriate type information.

Under native restrictions, calls to defineClass result in UnsupportedOperationException unless, it is used as a part of the class and interface resolution of the aforementioned instructions

\begin{enumerate}
    \item load class is blocked once the main has been initialised in ClassLoader\#loadClass
    \item in normal lambda uses invokedynamic
    \item in user called lambda metafactory does not use invokedynamic
    \item block system\#defineClass0 and 1 ?
    \item need to block: \begin{enumerate}
        \item java.lang.ClassLoader\#loadClass(java.lang.String)
        \item java.lang.invoke.LambdaMetafactory\#metafactory uses System\#defineClass0
        \item java.lang.invoke.MethodHandles.Lookup\#defineClass uses System\#defineClass0
        \item java.lang.invoke.MethodHandles.Lookup\#defineHiddenClass and defineHiddenClassWithClassData uses System\#defineClass0
        \item java.lang.invoke.MethodHandles.Lookup\#makeHiddenClassDefiner called from:
        \begin{enumerate}
            \item MethodHandles.Lookup\#defineHiddenClass and defineHiddenClassWithClassData
            \item InnerClassLambdaMetafactory
            \item InvokerBytecodeGenerator\#loadMethod (will be blocked later by reflection) called from:
            \begin{enumerate}
                \item InvokerBytecodeGenerator\#generateLambdaFormInterpreterEntryPoint
                \item InvokerBytecodeGenerator\#generateCustomizedCode -> forceInterpretation
                \item InvokerBytecodeGenerator\#generateNamedFunctionInvoker
            \end{enumerate}
            \item MethodHandleImpl.BindCaller\#makeInjectedInvoker (seems to work as is in NI)
        \end{enumerate}
        \item java.lang.reflect.Proxy.ProxyBuilder\#defineProxyClass uses System\#defineClass1 (called from asInstanceInterface and seems to work as is in NI)
    \item defineClass2 allowed as part of Class\#forName (will be restricted in reflection), restricted in all other cases -> does not support URLClassLoader
    \end{enumerate}
\end{enumerate}

\subsection{Resolving classes guarded by reflection}
Every call to defineClass that is guarded by reflection is simply funnelled through runtimeLoadClass.
The exception, is then classes loaded via loadClass. They must be pre-loaded (i.e they need to be included in the metadata and loaded before the main).

Transcript:\\
Vojin Jovanovic\\
I think we will have to funnel all of the ones guarded by reflection through "runtimeLoadClass".
Or we wait for that work to be done and load those classes when we see them in the config.
Saying because seems we will need all of the tracing etc. On that PR too.
Or as a separate PR.
Basically the loadClass needs to be tracked, if the class comes from the class path.

Capucine Berger-Sigrist\\
It looks like all the defineClass guarded by reflection already use runtimeLoadClass .

Vojin Jovanovic\\
Yes. Except the "loadClass" itself.
This should be the part of the spec.
That one needs reflection for that, i.e., that class will be pre-loaded.

Capucine Berger-Sigrist\\
I am still unsure what exactly pre-loading the class would mean (or does it just mean that it's in the metadata?).\\

Vojin Jovanovic
Well, if it is in the metadata we load it before it is not allowed.
---> So pre-loading would mean loading the class before we start checking (before the main is loaded), if and only if it's in the metadata.
 
%%%%%%%%%%%%%%%%
\section{Reflection}
%%%%%%%%%%%%%%%%
On MethodHandles:

When a method handle to a virtual method is invoked, the method is always looked up in the receiver (that is, the first argument).

A non-virtual method handle to a specific virtual method implementation can also be created. These do not perform virtual lookup based on receiver type. Such a method handle simulates the effect of an invokespecial instruction to the same method.
\url{https://docs.oracle.com/javase/8/docs/api/java/lang/invoke/MethodHandle.html#:~:text=Method%20handle%20creation,findStatic%20.} in Method handle creation

On Lookup factory methods:
The factory methods on a Lookup object correspond to all major use cases for methods, constructors, and fields. Each method handle created by a factory method is the functional equivalent of a particular bytecode behavior. (Bytecode behaviors are described in section 5.4.3.5 of the Java Virtual Machine Specification.)
\url{https://docs.oracle.com/javase/8/docs/api/java/lang/invoke/MethodHandles.Lookup.html}
5.4.3.5:
To resolve an unresolved symbolic reference to a method type, it is as if resolution occurs of unresolved symbolic references to classes and interfaces (§5.4.3.1) whose names correspond to the types given in the method descriptor (§4.3.3).
\url{https://docs.oracle.com/javase/specs/jvms/se21/html/jvms-5.html#jvms-5.4.3.5}

For reflection under native restrictions we introduce an additional step to method resolution as described in 5.4.3.5. Before any resolution can be done, the metadata must be checked to verify if the reflectively accessed element is in the closed-world.

